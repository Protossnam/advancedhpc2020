\documentclass[10pt, a4paper]{article}
\usepackage[utf8]{inputenc}
\usepackage[left=2cm, right=2cm, top=0.5in, bottom=0.5in, includefoot, headheight=13.6pt]{geometry}

\usepackage[utf8]{inputenc}
\usepackage{graphicx}
\usepackage{float}
\usepackage{titling}

\usepackage{minted}
\usepackage{xcolor,listings}
\usepackage{textcomp}
\usepackage{color}
\usepackage{scrextend}

\usepackage{tikz}
\usepackage{pgfplots}

\usepackage[caption = false]{subfig}

\usepackage[square, comma, numbers, sort&compress]{natbib}
\usepackage{graphicx, color}

\newcommand{\subtitle}[1]{%
  \posttitle{%
    \par\end{center}
    \begin{center}\LARGE#1\end{center}
    \vskip0.5cm}%
}

\usepackage{hyperref}
\usepackage{listings}

\title{[MI3.04a] Advanced Programming for HPC}
\subtitle{Report Get to know your GPU \\ Labwork 2}
\author{HUYNH Vinh Nam \\ M19.ICT.007}
\date{November 2020}

\renewcommand{\baselinestretch}{1.5} 

\begin{document}
% Report title
\maketitle

% Report structure

\section{Screenshot}
    \begin{center}
        \begin{figure}[H]
            \centering
            \subfloat[Original Image]{\includegraphics[scale=0.9]{images/GPU.png}} 
            \label{fig:gpu-devices}
        \end{figure}
    \end{center}
    
\clearpage

\hfill \hfill
\section{Bash output}
    \begin{minted}{bash}
./labwork 2
USTH ICT Master 2018, Advanced Programming for HPC.
Warming up...
Starting labwork 2
Number total of GPU : 2

GPU info
  Name:                    Tesla K40c
  Clock rate:              745000
  CUDA cores:              2880
  Multiprocessors:         15
  Warp size:               32
  Memory Clock rate (KHz): 3004000
  Memory Bus width (bits): 384
GPU info
  Name:                    GeForce GTX TITAN Black
  Clock rate:              980000
  CUDA cores:              2880
  Multiprocessors:         15
  Warp size:               32
  Memory Clock rate (KHz): 3500000
  Memory Bus width (bits): 384
labwork 2 ellapsed 0.8ms    
    \end{minted}

\clearpage

\hfill \hfill
\section{Implementation}

\begin{minted}{C++}
void Labwork::labwork2_GPU() {
    int nDevices = 0;
    // get all devices
    cudaGetDeviceCount(&nDevices);
    printf("Number total of GPU : %d\n\n", nDevices);
    for (int i = 0; i < nDevices; i++){
        // get informations from individual device
        cudaDeviceProp prop;
        cudaGetDeviceProperties(&prop, i);
        // something more here
        printf("GPU info\n");
        printf("  Name:                    %s\n", prop.name);  
        printf("  Clock rate:              %d\n", prop.clockRate);
        printf("  CUDA cores:              %d\n", getSPcores(prop));
        printf("  Multiprocessors:         %d\n", prop.multiProcessorCount);
        printf("  Warp size:               %d\n", prop.warpSize);
        printf("  Memory Clock rate (KHz): %d\n", prop.memoryClockRate);
        printf("  Memory Bus width (bits): %d\n", prop.memoryBusWidth); 
    }

}
\end{minted}

\end{document}