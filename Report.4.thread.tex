\documentclass[10pt, a4paper]{article}
\usepackage[utf8]{inputenc}
\usepackage[left=2cm, right=2cm, top=0.5in, bottom=0.5in, includefoot, headheight=13.6pt]{geometry}

\usepackage[utf8]{inputenc}
\usepackage{graphicx}
\usepackage{float}
\usepackage{titling}

\usepackage{minted}
\usepackage{xcolor,listings}
\usepackage{textcomp}
\usepackage{color}
\usepackage{scrextend}

\usepackage{tikz}
\usepackage{pgfplots}

\usepackage[caption = false]{subfig}

\usepackage[square, comma, numbers, sort&compress]{natbib}
\usepackage{graphicx, color}

\newcommand{\subtitle}[1]{%
  \posttitle{%
    \par\end{center}
    \begin{center}\LARGE#1\end{center}
    \vskip0.5cm}%
}

\usepackage{hyperref}
\usepackage{listings}

\title{[MI3.04a] Advanced Programming for HPC}
\subtitle{Threads \\ Labwork 4}
\author{HUYNH Vinh Nam \\ M19.ICT.007}
\date{November 2020}

\renewcommand{\baselinestretch}{1.5} 

\begin{document}
% Report title
\maketitle

% Report structure

\section*{Original Input}
    \begin{center}
        \begin{figure}[H]
            \centering
            \subfloat[Original Image]{\includegraphics[scale=0.25]{images/hl3.jpg}}
            \label{fig:cuda-input}
        \end{figure}
    \end{center}
    
\section*{Output}
    \begin{center}
        \begin{figure}[H]
            \centering
            \subfloat[Output]{\includegraphics[scale=0.25]{images/labwork4-gpu-out-hl3.jpg}}
            \label{fig:cuda-outputs}
        \end{figure}
    \end{center}

\hfill
\section*{Implementation}

\begin{minted}{C++}
__global__ void grayscale_2d(uchar3 *input, uchar3 *output, int img_width, int img_height) {
    int col = threadIdx.x + blockIdx.x * blockDim.x;
    int row = threadIdx.y + blockIdx.y * blockDim.y;
    if (col >= img_width || row >= img_height) return;
    int tid = row * img_width + col;
    output[tid].x = (unsigned char)(((int)input[tid].x + (int)input[tid].y 
                                   + (int)input[tid].z) / 3);
    output[tid].z = output[tid].y = output[tid].x;
}

void Labwork::labwork4_GPU() {
    // Calculate number of pixels
    int pixelCount = inputImage->width * inputImage->height;
    char *hostInput = inputImage->buffer;
    outputImage = static_cast<char *>(malloc(pixelCount * 3));

    // Allocate CUDA memory    
    uchar3 *devInput;
    uchar3 *devOutput;
    cudaMalloc(&devInput, pixelCount * 3);
    cudaMalloc(&devOutput, pixelCount * 3);
    
    // Copy CUDA Memory from CPU to GPU
    cudaMemcpy(devInput, hostInput, pixelCount * 3, cudaMemcpyHostToDevice);

    // Processing
    dim3 blockSize = dim3(32,32);
    dim3 gridSize = dim3((int)((inputImage->width + blockSize.x - 1) / blockSize.x),
                         (int)((inputImage->height + blockSize.y - 1)/ blockSize.y));
    grayscale_2d<<<gridSize, blockSize>>>(devInput, devOutput, 
                                          inputImage->width, inputImage->height);

    // Copy CUDA Memory from GPU to CPU
    cudaMemcpy(outputImage, devOutput, pixelCount * 3, cudaMemcpyDeviceToHost);


    // Cleaning
    free(hostInput);
    cudaFree(devInput);
    cudaFree(devOutput);
}

\end{minted}

\end{document}